% !TEX root = document.tex

Programmers spend significantly more time trying to comprehend existing code than writing new code.
They gain an understanding of the code by navigating the code base in an IDE, by reading documentation online, and by browsing code repositories on websites such as GitHub.
Creating rich experiences for a variety of programming languages across those various media is a large effort.
This effort might be worthwhile for popular languages, but for niche or experimental languages the required effort is often too large.
Solutions to reduce this effort in IDEs exist, like LSP, but to reduce the effort in other places, we introduce the \emph{Codex metadata format}, separating the language-specific generation of code metadata from its language-agnostic presentation.
We demonstrate this approach by implementing five language-specific generators (from LSP, CTAGS, TextMate, Agda and Elaine) and two language-agnostic presentations (PDF documents, and a code viewer website) of code and metadata.
To demonstrate different kinds of code metadata, we implemented four different code exploration services: syntax coloring, code navigation, structure outline, and diagnostic messages.
As part of this thesis, we submitted a paper in which we demonstrate that the Codex format successfully decouples languages and their metadata from their presentations.
In this thesis, we continue the work from that paper, discussing each topic more in-depth and expanding on ideas from the paper.
