% !TEX root = document.tex


\chapter{Editor Services}
\label{chap:editor-services}

Although programmers may have started out manually punching their programs into punchcards,
or even changing the wiring of the computer just to program it, few people still do this nowadays.
Instead, programmers write computer programs for a computer, on a computer.
The obvious advantage of that is of course that it is much easier to change the program after
it was written the first time.

Early on, there was \texttt{ed}~\autocite{ed}, short for `editor'.
\texttt{ed} was one of the standard tools included on all unix operating systems, and can still be used on many modern computers.
However, it has also been called ``the most user-hostile editor ever created'' (by Peter H. Salus) because of the
minimal visual feedback it gives.
For all kinds of different errors, \texttt{ed} simply prints a \texttt{?} without further context.

Since then, a lot has changed, and editors have become significantly more advanced.
Not just by actually allowing users to see the document they are editing, a feature which \texttt{ed} notably lacked,
but also by actively helping programmers to write their program.
For example, colouring different parts of a program based on their syntactical meaning, and completing words and lines based on context.
To provide these features, te editor will have a certain amount of knowledge about the language that is being edited, such as the language's grammar.

However, some editors know a lot more about the language that is being written than just the grammar.
A modern \ac{IDE} may know about different libraries you might want to use and their documentation,
work for multiple programming languages in large projects, and even automatically generate whole pieces of code based on context.
As well as simplifying writing code, an \ac{IDE} can also help with compiling and running code, as well as provide tooling to interactively debug it.
All these tools, which make it easier to write code, are called editor services, a term coined by \textcite{KatsKV08}

\section{An overview of editor services}\label{sec:an-overview-of-editor-services}

Different editors provide different editor services.
Nevertheless, the kinds of services provided overlap significantly, meaning we can try to list the main classes of services.
In ``The State of the Art in Language Workbenches'',~\autocite{ErdwegSV13} provides a feature model of language workbenches
\footnote{More will be said about language workbenches in ...}.
When \citeauthor{Fowler2004} first wrote about language workbenches, the relation between language workbenches and \acp{IDE} was already discussed~\autocite{Fowler2004}.
Hence, the feature model by \citeauthor{ErdwegSV13} also describes editor services as one of six main features of a language workbench.
In the feature model, editor services are further subdivided into

\begin{itemize}
    \item Syntactic services
    \begin{itemize}
        \item Highlighting
        \item Outline
        \item Folding
        \item Syntactic completion
        \item Diff
        \item Auto formatting
    \end{itemize}
    \item Semantic services
    \begin{itemize}
        \item Reference resolution
        \item Semantic completion
        \item Refactoring
        \item Error marking
        \item Quick fixes
        \item Origin tracking
        \item Live translation
    \end{itemize}
\end{itemize}

The feature model by \citeauthor{ErdwegSV13} was adapted by \citeauthor{Pelsmaeker2018} into a list of just editor services~\autocite{Pelsmaeker2018}.
This derived list further splits the services into more categories, as well as renaming a few to sound more intuitive.
For example, syntax highlighting was renamed to syntax colouring as highlighting would imply a change in background colour.
We also annotated each service in this list with a short explanation, not present in the original list by \citeauthor{Pelsmaeker2018}:

\begin{itemize}
%    \setlength\itemsep{-1em}
    \item Editor:
    \begin{itemize}
%        \vspace{-2\topsep}
%        \setlength\itemsep{-0.3em}
        \item Syntax colouring: giving different colours to different syntactical elements
        \item Code folding: hiding blocks of related code to reduce visual clutter
        \item Code completion: suggesting word and line completions based on context
    \end{itemize}
    \item Navigation and structure:
    \begin{itemize}
%        \vspace{-2\topsep}
%        \setlength\itemsep{-0.3em}
        \item Structure outline: visualizing the structure of code to quickly navigate it
        \item Reference resolution: providing quick navigating between related code
    \end{itemize}
    \item Help and documentation:
    \begin{itemize}
%        \vspace{-2\topsep}
%        \setlength\itemsep{-0.3em}
        \item Documentation: showing documentation of written elements in-editor
        \item Signature help: showing previews of an items parameters after typing the item
    \end{itemize}
    \item Actions and transformations:
    \begin{itemize}
%        \vspace{-2\topsep}
%        \setlength\itemsep{-0.3em}
        \item Automatic formatting: rewriting source code based on certain formatting rules
        \item Rename refactoring: language and context aware search-and-replace.
              May sometimes be called structural search and replace~\autocite{jetbrains_ssr}.
        \item Code actions: language specific actions that typically fix or transform code~\autocite{lsp_code_actions}
    \end{itemize}
    \item Life cycle
    \begin{itemize}
%        \vspace{-2\topsep}
%        \setlength\itemsep{-0.3em}
        \item Diagnostic messages: showing warnings and errors in the source text
        \item Debugging: stepping through code as it runs, allowing inspection of variables
        \item Testing: running tests directly from the editor, showing their success or failure
    \end{itemize}
\end{itemize}

\section{Providing editor services}\label{sec:providing-editor-services}

To provide editor services, the editor needs some information about the language it is providing services for.
The amount of information, depends on what services are provided.
The way these features are provided differs per language and editor.

\todo{
    if relevant, provide an example of for example how rust-analyzer and maybe another editor service provide do this.
    In any case mention how in many cases, editor services sort of integrate with a compiler since some services you want in an editor are also those you want in a compiler
    Also talk about editor services being dynamic
}


\section{Porting editor services}\label{sec:porting-editor-services}

Because editors need knowledge about a language to provide editor services for that language, not even the most advanced editor will support all languages.
Especially \acp{DSL} often have very little native support.
That is why many editors provide a way to be extended by users, often in the form of a plugin system.
That way, when a new language is created, users can add support for that language to the editor instead of the authors of the editor.

This leads to a complicated problem, called the \ac{IDE} portability problem~\autocite{KeidelPE16}.
The way every editor provides editor services is slightly different.
This means that supporting $m$ languages in $n$ editors requires $m \times n$ editor plugins, one for each
combination of a language and editor.
This issue often manifests itself as a need for language authors to create different editor plugins for every major editor.

In ``Portable Editor Services'', Pelsmaeker~\autocite*{Pelsmaeker2018} investigates this issue, and proposes a solution: \ac{AESI}.
\ac{AESI} is a system to create editor plugins.
\todo{Briefly explain AESI, then talk about LSP itself trying to solve the problem, and language workbenches which generate editor services based on a language's definition}


