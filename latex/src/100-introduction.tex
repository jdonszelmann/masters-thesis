% !TEX root = document.tex

\chapter*{Introduction}\label{chap:introduction}
\addcontentsline{toc}{chapter}{Introduction}

As programmers, we spend much more time reading code than writing it.
We try to get acquainted with a code base as part of our new job,
work our way though the API documentation of the new cutting-edge framework that everyone uses,
or attempt to comprehend code written decades ago.
All in all, compared to writing code, we spend an estimated ten times more time reading code~\autocite{clean_code}.

We read code in our highly interactive code editors and \acp{IDE},
but also explore code repositories on GitHub,
browse API documentation and specifications,
look for answers to our questions on StackOverflow,
and even internalize code from offline paper-based publications.

In this thesis, we introduce the term \emph{code exploration} to refer to this; the process of analyzing and understanding a software code base by examining its structure, components, and dependencies.
To explore code effectively, we use what we will call \emph{code exploration services}.
These form a subset of the more well-known \emph{editor services}~\autocite{ErdwegSV13}, but exclude services that are only useful when writing code.

Code exploration services range from simple yet effective syntax coloring, all the way to interactive services such as code navigation and hover information.
These services are not only useful in a code editor, but also in other \emph{code exploration media}; places where code might be explored, such as on documentation websites and in books.
The code exploration services and their presentations need to be adjusted to meet the limitations of the specific medium.
For example, syntax coloring is feasible in a printed book, but one cannot instantly jump to a definition.

There are existing solutions that facilitate code exploration.
Most major programming languages have spent effort to implement their own editor plugins that provide syntax coloring and additional editor services.
To support authors of code libraries, more and more languages include tooling to generate language-specific documentation websites from a code base.
For example, Rust has RustDoc, Haskell has Haddock, and Agda can output code as rich HTML and even LaTeX code.
However, all these solutions are very narrowly applicable to only particular languages where the developers have gone through the trouble of providing, implementing, and maintaining these services.
Especially for languages with less development power, investing in the implementation of such services may not be worthwhile.

At the same time, there are tools that apply to multiple languages but target only a very specific code exploration service.
For example, there is Bootlin's Elixir cross referencer,
which allows users to navigate C and C++ source code online, such as the Linux Kernel source code\footnote{Explore the Linux Kernel source code at \url{https://elixir.bootlin.com}}.
To provide code navigation in editors that support it, there is also CTAGS~\autocite{ctags}, which is a tool that generates an index of all identifiers found in a code base and which supports more than a hundred languages.
We will call these \emph{narrow} tools: those that either provide few services for many languages, or many services for few languages.

This hints at a problem often called the \emph{\ac{IDE} portability problem}, where $m$ languages providing services for $n$ editors require \problem{\times} implementations and related effort~\autocite{KeidelPE16}.
One solution to the \ac{IDE} portability problem is a system that \emph{decouples} editors from language-specific providers of editor services, which is what a system like \ac{LSP} does.
All languages that provide a \emph{language server} implementation can provide editor services to all editors with an \ac{LSP} client.
With \ac{LSP}, $m$ languages need and $n$ editors need to support \ac{LSP}, transforming the effort required from \problem{\times} into \problem{+}.

\todo{Why is LSP not good enough for code exploration services}

% Our prototype
% =============
% - We made a prototype that shows:
% - two approaches to generate metadata (from existing tools and from directly the compiler/static analysis tool)
% - two viewers
% - the intermediate format at work??

%%%%%%%%%%%%%%
The IDE portability problem forms the core of what this thesis is about.
There are maybe a dozen major code editors, but a nearly limitless number of places, such as many different websites, where having better code exploration services makes sense.
The \problem{\times} problem is thus even bigger for code exploration services than for editor services!
In this thesis, we will therefore aim to answer the question:

\begin{center}
    \textbf{Can we provide code exploration services outside editors in a way that decouples programming languages from code exploration media?}
\end{center}

\section*{Codex}

To answer that question, we present \emph{Codex}: an intermediate format for describing code base metadata, a solution to the \problem{\times} problem for rich code exploration.
The format is language-agnostic and can be extended to support new kinds of metadata for future code exploration services.
Thanks to its offline nature, the format allows the code to be explored at a point later in time from when the metadata is generated, even when the specific versions of tooling that were used on the code base are no longer available.
We discuss the reasoning that led us to the Codex format, describe what choices we made in the progress, demonstrate our implementation of
five prototype \emph{generators} of the Codex format and two different rich prototype \emph{presentations} of the Codex format and finally reflect on our experience making these prototypes. %

\paragraph{Contributions}

In summary, our contributions are:
\begin{itemize}
    \item the Codex metadata format, a tractable solution to the \problem{\times} problem for code exploration services, suitable for describing code bases of both DSLs and mainstream languages;
    \item a proof of concept comprising four prototype generators of Codex metadata (LSP, CTAGS, TextMate, Elaine), and two prototype code presentations derived from Codex metadata (LaTeX, HTML);
    \item an analysis of the trade-offs when designing a code metadata format.
\end{itemize}

These contributions align with the contributions of a paper which we submitted to Software Language Engineering (SLE) 2023 called \citetitle{our_paper}~\autocite{our_paper}.
Unfortunately, this paper was not accepted, the main reason being how we treated related work.
This thesis will reiterate many points from that paper, and is intended to serve as an extension of that paper, containing much more detailed related work and reasoning behind decisions.
Therefore, besides the contributions of that paper, in this thesis we present:

\begin{itemize}
    \item a much more thorough introduction of editor services and the IDE portability problem than was given in the paper;
    \item rigorous definitions of the terms ``Code Exploration'' and ``Code Exploration Services'', which we introduced;
    \item a more detailed study of existing tools that provide Code Exploration Services and how they relate to the Codex metadata format;
    \item a new generator that generates Codex metadata from Agda.
\end{itemize}

We do this in the following order.
We start in \cref{chap:editor-services} with an overview of what editor services are, together with a detailed explanation of the \ac{IDE} portability problem.
This leads into \cref{chap:code-exploration-services}, where we define the terms \emph{code exploration}, and \emph{code exploration services}
We relate these terms to what we learned about editor services in the previous chapter, go over existing solutions for code exploration services.
With this theory behind us, we will demonstrate our solution: the Codex Metadata Format in \cref{chap:codex}
<WIP> Finally, we discuss future work in \cref{chap:future_work} and conclude this thesis in \cref{chap:conclusion}.

%\todo[inline]{text-form TOC (commented below)}
%We give a high-level explanation of the Codex format in~\cref{sec:our-solution}, and demonstrate a proof of concept in~\cref{sec:demonstration} to show that our idea can indeed power such rich code exploration services.
%This demonstration is followed by an extensive discussion in~\cref{sec:considerations} of all the considerations that went into this proof of concept, followed by an analysis of its limitations and how we envision its future in~\cref{sec:fw}.
%We finish with a comparison of our solution to related work in~\cref{sec:prior-work}, and end with our conclusion in~\cref{sec:conclusion}.

% \paragraph {Background}
% \begin{itemize}
%     \item Editor Services
%     \item Code Exploration Services
% \end{itemize}
% \paragraph{Out of scope}
% \begin{itemize}
% \item speed of generation
% \end{itemize}
% \paragraph{Problem statement}
