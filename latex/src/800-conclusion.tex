% !TEX root = document.tex
\chapter{Conclusion}\label{chap:conclusion}

\todo[inline]{this is literally from the paper, so this should definitely be changed}

We presented Codex, a language-agnostic format for describing the metadata of a code base, with the purpose of providing code exploration services in different types of presentations, such as HTML websites and LaTeX documents.
The format decouples generating the metadata from its presentation, addressing  the \problem{\times} problem for code exploration services.
The format allows the code to be explored at a point later in time from when the metadata is generated, even when the specific versions of tooling that was used on the code base is no longer available.

By implementing multiple language-specific generators, we showed that the Codex metadata format can be generated from several existing narrow tools (e.g., TextMate grammars, LSP, CTAGS) for various programming languages (e.g. Rust, Haskell).
This includes generating metadata for a Domain-Specific Language (DSL), called Elaine, demonstrating that providing code exploration services for such languages without pre-existing tooling is feasible.
The code examples in the digital version of this paper are interactive, allowing the reader to navigate between code usages and definitions.
These interactive features and the syntax highlighting are generated based on the Codex metadata format, using our language-agnostic LaTeX presentation.
Additionally, we demonstrated an interactive HTML presentation that can be used to explore a code base.
We show that the Codex format successfully decouples languages and their metadata from their presentations.