% !TEX root = document.tex

\chapter{Code Exploration Services}
\label{chap:code-exploration-services}

As the name implies, a programmer's job generally involves writing programs.
However, a vital part of writing code is also reading and exploring code.
In the book ``Clean Code: A Handbook of Agile Software Craftsmanship''~\autocite{martin_reading_code_ratio},
the author estimates that a programmer may spend up to ten hours reading code for every hour of time they spend writing code.
\todo{what's the goal of editor services?}

Some examples of such tasks that do not necessarily involve writing code might be reading code from last week, or code written by colleagues.
Especially in teams, programmers also commonly review code, which mainly involves reading code and verifying its correctness.
Furthermore, programmers may read code on websites such as \url{https://stackoverflow.com} and \url{https://github.com},
and in documentation, to learn from others about how to solve a problem.
All these interactions programmers might have with code, that do not involve writing code, we will call code exploration from now on.

% Reference https://www.informit.com/articles/article.aspx?p=1235624&seqNum=5?
% Reference https://www.youtube.com/watch?v=QoZU2yN8caY

Exploring programs in an editor can be easier than in other places, because many editors provide editor services that simplify exploring and understanding code.
Still, editors aren't the only places where programmers explore code.
The main other locations being on websites, in PDFs (for example, a paper), and in books.
Although these places are used for exploring code, they often provide few services that help programmers to do so like editors would.
It might be that only syntax colouring is available, for example.
However, we think that those services traditionally found in editors, might actually be useful when reading code in other places besides editors
Those services, which are useful outside an editor, we call code exploration services.

As an example, and a notable exception to websites only providing syntax colouring, there is Github\footnote{\url{https://github.com}}.
GitHub, apart from being a git server where programmers store their code, is also a place where lots of code is read.
For example, as a part of code reviews.
To aid programmers, GitHub has been getting more and more features that make exploring code simpler such
as more advanced search functionality, and recently code navigation.
The latter is only supported for a few specific programming languages though\footnote{\url{https://docs.github.com/en/repositories/working-with-files/using-files/navigating-code-on-github}}.

This does raise the question: what other services might (in general) be useful code exploration services?

\section{An overview of code exploration services}\label{sec:an-overview-of-code-exploration-services}

In chapter~\ref{chap:editor-services}, we discuss a list of important editor services.
For each, we will discuss whether they might be useful as code exploration services as well.
Before we can do that however, we first need to discuss when a code exploration service is useful.
We think that a code exploration service is useful, if it has the potential to help a programmer understand the program they are reading quicker.
With that definition, how useful is each the editor services as code exploration services?

\subsection*{Syntax colouring}

The usefulness of syntax colouring for code comprehension has been extensively researched.
Although one study, with a low sample size, seems to claim that colouring has a positive effect for novices, but not for
more experienced programmers~\autocite{Sarkar15a-0}, there is also work that concludes that the effect is in fact minimal.
In a study where eye movements of participants were tracked, no significant difference between black-and-white code and coloured code\autocite{beelders2016syntax}.
Similarly, novice students would not perform better at programming tasks where syntax was coloured compared to tasks where syntax was not coloured~\autocite{HannebauerHG18}.

Although there seems to be relatively little evidence that syntax colouring actually helps programmers, it is what many programmers will expect when reading code.
All editors evaluated by \citeauthor{Pelsmaeker2018} support syntax colouring to a certain extent~\autocite{Pelsmaeker2018},
and one could argue that the coloured text distinguishes code from the surrounding content.
Therefore, we would argue that syntax colouring can be useful when exploring code.

\todo{do I want full cited explanations of each of this lke above, or is that not really important for the research on file formats and should I just state that
these are the things we're going to look at}
\subsection*{Code folding}

\subsection*{Code completion}

Code completion only helps programmers write code, and has little to do with code exploration.

\subsection*{Structure outline}

\subsection*{Reference resolution}

\subsection*{Documentation}

\subsection*{Signature help}

\subsection*{Automatic formatting}

\subsection*{Rename refactoring}

\subsection*{Code actions}

\subsection*{Diagnostic messages}

\subsection*{Debugging}



\subsection*{Testing}

%As demonstrated by Github, it certainly seems like it is possible to provide more services to the readers of programs.
%One might call such services for code readers, code exploration services.
%However, that's not quite

\subsection{Code search}\label{subsec:code-search}

Although the list of code exploration services has its roots in a list of editor services, which itself has its roots in
frequently cited theory by \citeauthor{ErdwegSV13}, we actually think that one feature may be missing from the original list.
That feature is search.
That search is missing is interesting, because the feature is pretty much universally supported by editors, even `dumb' ones.
Even many websites, where we showed that often relatively few services are supported, offer advanced search options.

So why is search missing?


\section{Availability of code exploration services}\label{sec:availability-of-code-exploration-services}

\todo{Survey of other websites, showing that they pretty much only support syntax colouring}



\subsection{Github}
\subsection{Gitlab}
\subsection{MdBook}
\subsection{StackOverflow}
\subsection{Reddit}
\subsection{Latex and PDF}
\subsection{Bootlin}
\subsection{Monaco}


%However, to take advantage of editor services, a programmer needs to download the sourcecode to their own computer and actually
%open the program in an editor, which can be inconvenient.

% Sometimes you look at code not in an IDE (like online, or in papers)
% What services are available in those places?
% Most places provide syntax colouring
% Some places provide code navigation (Github, cross referencers)
% However for very limited cases (only some languages)
% What's different from IDEs to these places where code is read? Well,
% What services could theoretically be provided?
% We call these services code exploration services


%jupyter
%mdbook